
\documentclass[a4paper]{article}
%\raggedright
\usepackage[12pt]{extsizes}
\usepackage[utf8]{inputenc}
\usepackage[T1]{fontenc}
\usepackage[default,oldstyle,scale=0.95]{opensans} %% Alternatively
\usepackage[french]{babel}
\usepackage[hmargin=2cm,vmargin=3cm]{geometry}
\usepackage{graphicx}
\usepackage{hyperref}
\usepackage{media9}
%\usepackage{tabularx}
%\usepackage{fontspec}
%\setmainfont{Open Dyslexic}[WordSpace={2.5,1.2,0}]


% set default figure placement to htbp
\makeatletter
\def\fps@figure{htbp}

\makeatother
\renewcommand{\baselinestretch}{1.5} %interline
\setlength{\parskip}{1em} %interparagraphe
\title{Les échanges sur le web et l'architecture client-serveur}
\date{2020-2021}
\author{Rémi Pépin}


\begin{document}
\maketitle

\hypertarget{header-n0}{%
\section{Les échanges sur le web et l'architecture
client-serveur}\label{header-n0}}

Ce document couvre les connaissances de bases que vous êtes sensé avoir
à la fin de ce cours sur des notions de base d'internet et du web et de
l'échange de données sur le web. Comme tout outil, savoir comment il
fonctionne vous permettra de mieux l'utiliser et de savoir ce qui est
possible et impossible avec.

\includemedia[
width=0.6\linewidth,height=0.3375\linewidth, % 16:9
activate=pageopen,
flashvars={
	modestbranding=1 % no YT logo in control bar
	&autohide=1 % controlbar autohide
	&showinfo=0 % no title and other info before start
	&rel=0 % no related videos after end
}
]{1594196295282.png}{http://www.youtube.com/v/r382kfkqAF4?rel=0}

\begin{quote}
 La lecture du document et des différentes vidéos devrait vous prendre
environ une heure et quart. C'est le temps passé l'an dernier pour
couvrir les même notions en cours. Des espaces pour prendre des notes
sont disponibles dans ce document si vous voulez y ajouter des
informations (il vous faudra par contre l'imprimer sous forme de pdf
pour conserver vos notes ce qui vous fera perdre les vidéos).

Une fois ce FOAD fait vous devrez tester vos connaissances avec un QCM
sur Moodle. Une note de 15/20 est attendue pour le valider. \textbf{S'il
n'est pas validé, un malus d'un point sera appliqué sur votre note du
module.}
\end{quote}

\hypertarget{header-n6}{%
\subsection{Le web}\label{header-n6}}

\hypertarget{header-n7}{%
\subsubsection{Internet et le web}\label{header-n7}}

Si dans la vie de tous les jours, l'internet et le web sont souvent
considérés comme des notions interchangeables, il n'en est rien, et le
web et internet sont bien deux choses différentes.

\begin{quote}
\textbf{Internet} : l'internet ou l'inter-connexion des réseaux, ou
réseau des réseaux désigne l'ensemble des infrastructures physiques et
protocoles logiques qui permettent l'inter-connexion des réseaux et
l'échange des données sans présupposer de leur type.

\textbf{Le web} : le web est une des applications d'internet, mais
elle n'est pas la seule. Le web désigne l'échange d'information décrite
dans des documents, historiquement des pages HTML
(\textbf{H}yper\textbf{T}ext \textbf{M}arkup \textbf{L}anguage). Mais ce
n'est pas la seule application du web. Les mails avec le SMTP
(\textbf{S}imple \textbf{M}essage \textbf{T}ransfer \textbf{P}rotocole,
les chats IRC (\textbf{I}nternet \textbf{R}elay \textbf{C}hat), le
\emph{peer-to-peer} avec BitTorrent sont d'autres services qui utilisent
internet.

Pour simplifier internet désigne les infrastructure dans lesquels les
données circulent alors que le web désigne les pages qui sont liées les
unes aux autres par des liens hypertexte et la manière de le récupérer.
Donc le web dépend internet, mais la réciproque est fausse.
\end{quote}

            Notes internet vs web

\hypertarget{header-n16}{%
\subsubsection{Le protocole HTTP et HTTPS}\label{header-n16}}

Le web est fondamentalement lié au protocole HTTP
(\textbf{H}yper\textbf{T}ext \textbf{T}ransfer \textbf{P}rotocol).

\begin{quote}
\textbf{Protocole} : un protocole informatique défini l'ensemble des
règles qui vont définir des échanges. Il normalise la communication et
donne un cadre à respecter pour rendre la communication possible entre
une multitude d'éléments différents car ils suivent les même règles.
\end{quote}

Le HTTP est un protocole qui fonctionne sur le principe de requête -
réponse. Un client demande une ressource à un serveur (c'est la
requête), et le serveur lui fournit s'il le peut (c'est la réponse).

Il faut bien comprendre que le web n'a jamais été prévu pour devenir ce
qu'il est aujourd'hui. Heureusement il a été bien conçu et a pu évoluer
pour passer d'un simple outil pour accéder à des documents au service le
plus utilisé au monde. Au début la sécurité n'était pas importante.
Quand on accède simplement à des papiers de recherche encrypter les
données n'est pas forcément une priorité. Par contre quand on veut
acheter des objets en ligne, communiquer de manière sécurisée, acheter
ou vendre des produits financier et que nos informations vont transiter
on ne sait où et sont donc potentiellement accessibles à tous, la
sécurité devient importante.

 

            Note http, https

Pour sécuriser les échanges le HTTP a dû évoluer pour devenir le HTTPS
(\textbf{H}yper\textbf{T}ext \textbf{T}ransfer \textbf{P}rotocol
\textbf{S}ecure). Le principe est simple, on garde le principe du HTTP
mais on chiffre les données échangé grâce à l'ajout d'une couche de
sécurité (SSL pour \textbf{S}ecure \textbf{S}ockets \textbf{L}ayer, ou
TSL pour \textbf{T}ransport \textbf{L}ayer \textbf{S}ecurity). La
théorie de la cryptographie est assez complexe, mais il n'y a pas besoin
d'être un expert du domaine pour comprendre comment le HTTPS fonctionne
et les choix de ce protocole.

Il existe deux grands types de chiffrement :

\begin{itemize}
\item
  Le symétrique ;
\item
  Et le asymétrique.
\end{itemize}

Dans les deux cas le principe général consiste \textbf{à prendre un
message le transformer pour le rendre impossible à comprendre pour
toutes personne qui n'a pas la clef de déchiffrement}. Un message
correctement crypté pourrait être affiché dans la rue sans que quiconque
à part les personnes qui dispose de la clef de déchiffrement ne puisse
le comprendre. Fondamentalement le cryptage consiste juste à rendre
illisible un message sauf pour quelques personnes. La clef de
déchiffrement peut être un objet physique, une connaissance spécifique
ou une fonction mathématique. Ce n'est donc pas un précédé informatique
par nature. Mais l'informatique excelle dans ce domaine car elle permet
de réaliser rapidement des calculs complexes.

\hypertarget{header-n34}{%
\paragraph{Chiffrement symétrique}\label{header-n34}}

Dans le chiffrement symétrique l'émetteur du message et le récepteur
dispose de la même clef qui leur permet à la fois de chiffrer et de
déchiffrer les données. On pourrait croire que le problème de ce
fonctionnement c'est que dans un système à \(n\) machines, il faut
\(\frac{n(n-1)}{2}\)clefs (donc un \(\Theta(n^2)\) clef). Mais en fait
non, ce n'est pas un réel problème. La grande difficulté se pose sur
l'échange des clefs. \textbf{En effet il faut s'assurer qu'on arrive à
échanger toutes ces clefs de manière sécurisée}. Car si une clef est
compromise entre deux machines, un attaquant va pouvoir lire les
communications et insérer ses propres messages entre elles. Si vous
voulez mettre ce genre de communication avec des amis vous pouvez vous
échanger des clefs USB, mais imaginez le calvaire si, quand vous voulez
accéder à un site sécurisé, vous deviez attendre une clef USB avec la
clef de chiffrement ! (en plus l'échange postal n'est pas le plus
sécurisé du monde). C'est clairement intenable et ne permet pas à des
milliards d'équipement dez communiquer entre eux.

\hypertarget{header-n36}{%
\paragraph{Chiffrement asymétrique}\label{header-n36}}

Le chiffrement asymétrique quand à lui permet de résoudre ce problème.
Au lieu d'avoir une clef entre émetteur et récepteur, le récepteur
dispose de deux clefs, une clef dite publique et une clef dite privée.
chacune de ces clefs a un rôle bien précis, \textbf{la clef publique
permet uniquement de chiffrer un message, alors que la clef privé va le
déchiffrer}. Cela permet de rendre la clef publique accessible à tous
car elle ne permet pas de déchiffrer des données. Ainsi l'émetteur va
demander au récepteur sa clef publique, chiffrer son message et lui
envoyer. Ce message peut être intercepté par tous sans risque car seul
le récepteur peut le déchiffrer (théoriquement). Par contre si le
récepteur veut répondre, il ne peut pas le faire de manière sécurisée.
En effet s'il chiffre la réponse avec sa clef publique, l'émetteur ne
pourra pas lire la réponse. Mais au moins on a un chiffrement dans un
sens.

On pourrait imaginer alors qu'émetteur et récepteur aient un couple clef
publique, clef privée, mais ce n'est pas la solution retenue pour le
HTTPS.

\hypertarget{header-n39}{%
\paragraph{Fonctionnement du HTTPS}\label{header-n39}}

La solution retenue pour le HTTPS est de faire un \textbf{échange en
deux temps, d'abord asymétrique puis symétrique}. Le client (que
j'appelais émetteur avant) va demander la clef publique su serveur
(récepteur). Le clef publique sera contenu dans le certificat du
serveur. Le certificat est en quelque sorte sa carte d'identité. Un
organisme tiers de confiance (autorité de certification) valide
l'identité du serveur. C'est la même chose avec une carte d'identité. Un
état sert d'organisme tiers en "validant" votre identité légale.
L'émetteur va aller vérifier auprès de l'autorité de certification si ce
certificat est véridique (pour éviter l'usurpation d'identité de
serveur).

Si l'autorité valide le certificat, le client va ensuite générer une
clef secrète et l'envoyer au serveur en la chiffrant avec la clef
publique. Ce message ne peut être déchiffré que par le récepteur, donc
il peut transiter tranquillement sur internet. Une fois qu'il a
déchiffré la clef générée, le serveur utilise la clef pour générer la
clef de session qui va servir à chiffrer les futurs échanges. Le client
va lui aussi déterminer la clef de session avec la clef secrète qu'il a
généré. À partir de maintenant les échanges seront chiffrés de manière
symétrique entre client et serveur.

\begin{quote}
Rassurer vous, tout ça est fait automatiquement par votre navigateur ou
les bibliothèque que vous allez utiliser. Mais il est important de
comprendre ce qu'il se passe, car par moment on peut aller sur des sites
où notre navigateur nous alerte sur le certificat du site (autosigné,
périmé). Cela n'est pas forcément grave quand on comprend le problème.
Dernière chose, un certificat est payant. Par extrêmement cher, mais pas
gratuit pour autant.
\end{quote}

\begin{figure}
\hypertarget{mermaid}{%
\centering
\includegraphics[width=7.02083in,height=\textheight]{1594196295282.png}
\caption{}\label{mermaid}
}
\end{figure}

            Notes chiffrement

\hypertarget{header-n47}{%
\subsubsection{URI, URL, URN et DNS}\label{header-n47}}

Maintenant que vous comprenez comment les données sont échangées, il
reste encore un gros mystère, comment les données sont localisées. En
effet pour un ordinateur www.ensai.fr ne le renseigne pas sur la
localisation de la ressource demandé. Bien que URL signifie
\textbf{U}niform \textbf{R}essource \textbf{L}ocator c'est surtout une
localisation essentiellement humaine, et non pas machine. Vous avez déjà
dû entendre parler d'adresse IP (\textbf{I}nternet \textbf{P}rotocol).
L'adresse IP est la vraie adresse de votre machine. L'URL est seulement
son nom \emph{human friendly} pour qu'on s'en souvienne facilement.
Votre ordinateur va avoir besoin d'un annuaire pour faire faire le lien
entre cette URL et l'IP de machine qui héberge la ressource. Ce système
appelle le DNS pour \textbf{D}omain \textbf{N}ame \textbf{S}ystem.

            Notes URL, URN, URI, DNS

\hypertarget{header-n54}{%
\subsection{Le modèle client serveur}\label{header-n54}}

\hypertarget{header-n55}{%
\subsubsection{L'architecture client-serveur}\label{header-n55}}

Le modèle \textbf{client-serveur est l'architecture qui domine le web
actuellement}. Son résumé est simple, des clients font des requêtes HTTP
à des serveurs qui y répondent. Les client fonts seulement des actions
de temps en temps, alors que le serveur doit pouvoir répondre à toutes
les requêtes qui lui arrive, jour et nuit. Ainsi la différence entre
client et serveur n'est finalement que liée au fonctionnement du modèle,
le client demande et le serveur répond. Ce qui fait que presque tout
équipement informatique peut être client et serveur en fonction du
contexte. Vous pouvez très bien utiliser une raspberry pie comme petit
serveur de contenu web ou comment client pour récupérer des données. Ou
vous pouvez héberger votre site web sur votre pc portable. Par contre
dés que votre pc s'éteint votre site devient inaccessible.

 

            Note client serveur

\hypertarget{header-n60}{%
\subsubsection{API/web services REST et requête http en
python}\label{header-n60}}

Maintenant que vous avez du comprendre la notion de client-serveur,
abordons la notion d'API ou de \emph{web services}. Il est important que
vous compreniez dans les grandes lignes comment ils fonctionnent car de
plus en plus de données sont échangées par leur biais.

Vous pouvez vous arrêter quand la vidéo parle de nodejs. C'est une
plateforme logicielle en JavaScript. Comme vous ne connaissez pas ce
langage la fin de la vidéo va vous sembler compliquée (pour info le
cours de visualisation des données et de technologie mobile du second
semestre vont vous faire découvrir le JavaScript, et dernière chose Java
et JavaScript n'ont rien à voir)

            Notes webservices

\hypertarget{header-n68}{%
\subsection{Un peu de culture informatique}\label{header-n68}}

Terminons ce FOAD par 3 petites vidéos sur des notions que vous avez
sûrement déjà entendu, à savoir:

\begin{itemize}
\item
  Proxy
\item
  VPN
\item
  Coockies
\end{itemize}

Ce sont des notions assez basiques que vous devez comprendre, pas pour
ce cours, mais pour devenir des personnes consciente du fonctionnement
d'internet et du web.

 

            Notes diverses

\end{document}
